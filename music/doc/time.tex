\section{Time components}
\comp{tempo}
{Changes the tempo of components feeding in}
{mod:float, 8$\times$in:flow}{flow:flow}
The tempo component modifies the tempo global variable, pushing
the original onto the stack. It then runs its inputs, and restores the
original. Thus, all components in the input will use the accelerated tempo,
affecting their timing if they use tempo-based times.
\startparams
tempo & float & the new tempo in bpm \\
mod & float & modulation amount ($out = p_{mod} \times in_{mod}$)\\
\end{tabularx}

\comp{clock}
{Gives a regular ``ticking'' integer output, relative to the tempo.}
{}{4$\times$tick:int, outputs 1+ happen after a delay}
Generates a regular tick (integer 1) on outputs, which are 0 the
rest of the time. The tick frequency is relative to the tempo,
and is set by parameters. Output 0 is the reference, the other outputs
tick on subsequent updates -- so output 0 ticks, then on the next
frame output 1 will tick and so on. If the \emph{gapbeats} parameter
is set, the interval between the outputs is increased to be some
fraction of the time between tempo beats.
\startparams
poweroftwo & int & tempo is multiplied by 2 to the power of this value\\
mul & float & tempo is then multiplied by this value\\
add & float & tempo finally has this value added\\
gapbeats & float & fraction of a beat between each output
\end{tabularx}

\clearpage
\comp{tickcombine}
{Combines multiple tick outputs into a single tick stream.}
{4$\times$in:int}{out:int}
Takes a number of inputs which are usually zero. If any of the
inputs are nonzero, the output will be 1. Typically used to combine
a number of clocks (or similar) into a single stream.

\comp{osc}
{Generates a periodic waveform.}
{mod:float}{wave:float}
Given a frequency in periods/beat (according to the current tempo),
produces a wave on the output. Various waveforms are available.
The phase can be modulated by an input.
\startparams
wave & enum & waveform\\
freq & float & frequency (periods/beat)\\
phase & float & phase\\
mod & float & amount of phase modulation\\
amp & float & amplitude of wave\\
offset & float & added to output\\
minzero & boolean & if true, wave's minimum value is zero (i.e. $amp/2$ is added)\\
width & float & pulsewidth if square wave selected (default is 0.5, which is a true square wave)
\end{tabularx}


\comp{time}
{Gives the time since the program started.}
{}{time:float}
Produces the number of seconds since the start of the program,
times a ``rate'' parameter.
\startparams
rate & float & the number of seconds is multiplied by this to
produce the output
\end{tabularx}

\clearpage
\comp{env}{Envelope generator}
{trigger:int}{out:float}
This is a 6 stage envelope generator. Each stage has a level
and a time relative to the start, in seconds. There is an implied
level=0, time=0 stage. The generator can function in three modes:
\begin{itemize}
\item \textbf{retrig}: the envelope restarts whenever \emph{trig} is
non-zero;
\item \textbf{noretrig}: the envelope only restarts when \emph{trig} is
high and any previous envelope has completed;
\item \textbf{freerun}: the envelope retriggers as soon as it stops
running, and starts running as soon as its output is connected to
a running component.
\end{itemize}
\startparams
mode & enum & retrigger mode as described above\\
rate powerof2 & int & the time since the start is multiplied by 2
to this power before being used to calculate the envelope state\\
env & env & the actual envelope (considered as a single parameter
so the editor can put a fancy graphic on it)
\end{tabularx}
Within the envelope parameter each stage has a level and time.
Times are autocorrected: a later stage cannot have its time moved
to before an earlier stage, and moving an earlier stage to after
a later stage will cause the later stage times to move.
Two stages may have the same time, in which case the higher-numbered
stages level is jumped to.
