\documentclass[a4paper]{article}
%\usepackage{wrapfig}
\usepackage[pdftex]{graphicx}
%\usepackage[version=3]{mhchem}
\usepackage{fancyvrb}
%\usepackage{multirow}
\usepackage{url}
\usepackage{amsmath}
\usepackage{amssymb}

\author{Jim Finnis}
\title{Music Components}

\newcommand{\todo}[1]{
    \begin{center}
    \fbox{\parbox{4in}{\textbf{To Do}\vspace*{1em} \\#1}}
    \end{center}}

\newenvironment{important}
{\begin{quote}\color{red}\large%
\begin{center}{\Large \textbf{IMPORTANT}}\end{center}}
{\end{quote}}

\newenvironment{notebox}[1][]
{\begin{framed}%
\ifthenelse{\equal{#1}{}}{}{%
    \begin{center}%
    {\Large \textbf{#1}}%
    \end{center}%
    }%
    }%    
{\end{framed}}

% different commands for left and right margin notes                                                                                                   
\newcommand{\weeboxL}[1]{\rule{15mm}{0.5pt}\\ \parbox[t]{15mm}{\raggedright\footnotesize#1}}                                                                         
\newcommand{\weeboxR}[1]{\rule{15mm}{0.5pt}\\ \parbox[t]{15mm}{\raggedleft\footnotesize#1}}
                                                                                                                                                       
% this selects which command to use for margin notes                                                                                                   
\newcommand{\marg}[1]{\marginpar[\weeboxL{#1}]{\weeboxR{#1}}}

\DefineVerbatimEnvironment{v}{Verbatim}{
    %numbers=left,numbersep=5pt,
    %frame=lines,framerule=0.5mm,
    fontsize=\small,xleftmargin=15pt}
\DefineVerbatimEnvironment{vv}{Verbatim}{
    %numbers=left,numbersep=5pt,
    %frame=lines,framerule=0.5mm,
    fontsize=\small,xleftmargin=15pt}
\DefineVerbatimEnvironment{v2}{Verbatim}{
    %numbers=left,numbersep=5pt,
    %frame=lines,framerule=0.5mm,
    fontsize=\scriptsize,xleftmargin=0pt}
\DefineVerbatimEnvironment{v3}{Verbatim}{
    %numbers=left,numbersep=5pt,
    %frame=lines,framerule=0.5mm,
    fontsize=\tiny,xleftmargin=0pt}
\DefineVerbatimEnvironment{bv2}{BVerbatim}{
    %numbers=left,numbersep=5pt,
    %frame=lines,framerule=0.5mm,
    fontsize=\scriptsize,xleftmargin=0pt}
\DefineVerbatimEnvironment{bv}{BVerbatim}{
    %numbers=left,numbersep=5pt,
    %frame=lines,framerule=0.5mm,
    fontsize=\scriptsize,xleftmargin=0pt}
\DefineVerbatimEnvironment{bvc}{BVerbatim}{
    %numbers=left,numbersep=5pt,
    %frame=lines,framerule=0.5mm
    commandchars=+\[\],
    fontsize=\scriptsize,xleftmargin=0pt}

\setcounter{secnumdepth}{5}
\setcounter{tocdepth}{5}
\usepackage[hmargin=3cm,vmargin=5cm]{geometry}
\usepackage{hyperref}
\usepackage{tabularx}
\usepackage{booktabs}
\hypersetup{colorlinks}

\usepackage{listings}
\usepackage[usenames,dvipsnames]{color}

\usepackage{array}


% \component{name}{desc}{inputs(Comma sep)}{outputs(comma sep)}
\newcolumntype{Z}{>{\raggedright\let\newline\\\arraybackslash\hspace{0pt}}r}
\newcommand{\comp}[4]{%
    \phantomsection
    \addcontentsline{toc}{subsection}{#1}
    \label{comp:#1}
    \vspace*{4em}
    \noindent \begin{tabularx}{\textwidth}{@{}XZ@{}}
    \toprule
    \textbf{\Large #1} & \emph{#2} \\
    \end{tabularx}
    \begin{tabularx}{\textwidth}{@{}XZ@{}}
    \textbf{in:} \emph{#3} & \textbf{out:} \emph{#4}   \\    
    \bottomrule
    \end{tabularx}\vspace{1em}
}

\newcommand{\startparams}
{\vspace*{1em} \leavevmode \\ \noindent \textbf{Parameters:} \vspace{1em}\\
\noindent \begin{tabularx}{\textwidth}{@{}lp{1in}X@{}}}
{\vspace*{1em} \leavevmode \\ \noindent \textbf{Parameters:} \vspace{1em}\\
\noindent \begin{tabularx}{\textwidth}{@{}lp{1in}X@{}}}

\begin{document}

\maketitle
\LetLtxMacro\oldincludegraphics\includegraphics
% Stuff to redefine includegraphics to show the document file
% underneath; useful for referencing.
\renewcommand{\includegraphics}[2][]{%
%    \rotatebox[origin=tr]{270}{\tiny #2} %
    \oldincludegraphics[#1]{#2}%
    \\{\scriptsize #2}%
}

\section{Introduction}
This document describes the various components available to the
Stumpy system using the ``music'' server, with the editor currently
codenamed VIOLET PHEASANT. It attempts to be divided into categories
as they appear in the editor, but discussion is also made of common
idioms.

Hopefully it might even be remotely up-to-date.






\section{Control components}
\comp{output}
{A ``root'' component which simply runs all its inputs}
{4$\times$input:flow}{}
The patch runs by finding all \emph{output} components and running
their inputs. Therefore, this component must be present for the patch to run. 

\comp{mixer}
{A component which simply runs all its inputs}
{6$\times$input:flow}{output:flow}
This component, when run, runs all its inputs.

\comp{switcher}
{Selects one input to pass through to the output}
{select:float, 6$\times$inputs:ANY}{output:ANY}
This component simply selects one of its inputs and executes it,
passing the obtained value to the output without interpreting it.
The output and inputs (apart from the selection value) may therefore
be of any type. 

The selection value is typically in the range [0,1], and is
mapped onto [0,$n$] where $n$ is the number of the highest connected
\emph{any} input. For example, if only two \emph{any} inputs are connected,
values less than 0.5 will select the first choice, otherwise the 
second choice will be selected.
\startparams
mul & float & select input is multiplied by this value before use \\
add & float & select input is then added to this value before use\\
unit & bool & determines whether or not the value is mapped onto [1,0]
or [0,$n-1$] for selection (true means the former)
\end{tabularx}

\clearpage
\comp{threshold}
{Outputs triggers when thresholds are crossed}
{input:float}{trigger$\times$6:int}
Contains 6 thresholding units, which detect when the input
crosses a boundary (rising or falling) and output a single pulse
trigger on the output for the unit; or alternatively a 1 or 0
depending on whether the value is below or above threshold.
\startparams
mul & float & input is multiplied by this value before use \\
add & float & input is then added to this value before use\\
&&\\
level $n$ & float & level of threshold\\
type $n$     & inactive, rising, falling, low, high& threshold type\\
\end{tabularx}


\comp{connectin}
{Connector (input)}
{input:ANY}{}
This is the input side of a connector, which can be used to
organise complex patches. The output comes out of an output
connector with the same number.


\comp{connectout}
{Connector (output)}
{output:ANY}{}
This is the output side of a connector, which can be used to
organise complex patches. The input comes from an input
connector with the same number.

\clearpage
\comp{crossfade}
{Crossfader}
{select:float, input$\times$6:flow}{output:flow}
This changes the amplitude of the inputs depending on which one
is selected by modifying the global amplitude. 
Inputs whose amplitude is nonzero will not run at all.
The select value determines which input is ``loudest''.
If the select value is an integer, only that input
will have a nonzero amplitude. If the select value is between
two inputs (it is a float), the two inputs either side will both
run, with their relative amplitudes dependent on how far between
them the select value is.
\[
amp(n) = \begin{cases}
1-|sel-n| & \quad\mathrm{if}\, 1-|sel-n|>0\\
0 & \quad\mathrm{otherwise}
\end{cases}
\]


\section{Time components}
\comp{tempo}
{Changes the tempo of components feeding in}
{mod:float, 8$\times$in:flow}{flow:flow}
The tempo component modifies the tempo global variable, pushing
the original onto the stack. It then runs its inputs, and restores the
original. Thus, all components in the input will use the accelerated tempo,
affecting their timing if they use tempo-based times.
\startparams
tempo & float & the new tempo in bpm \\
mod & float & modulation amount ($out = p_{mod} \times in_{mod}$)\\
\end{tabularx}

\comp{clock}
{Gives a regular ``ticking'' integer output, relative to the tempo.}
{}{4$\times$tick:int, outputs 1+ happen after a delay}
Generates a regular tick (integer 1) on outputs, which are 0 the
rest of the time. The tick frequency is relative to the tempo,
and is set by parameters. Output 0 is the reference, the other outputs
tick on subsequent updates -- so output 0 ticks, then on the next
frame output 1 will tick and so on. If the \emph{gapbeats} parameter
is set, the interval between the outputs is increased to be some
fraction of the time between tempo beats.
\startparams
poweroftwo & int & tempo is multiplied by 2 to the power of this value\\
mul & float & tempo is then multiplied by this value\\
add & float & tempo finally has this value added\\
gapbeats & float & fraction of a beat between each output
\end{tabularx}

\clearpage
\comp{tickcombine}
{Combines multiple tick outputs into a single tick stream.}
{4$\times$in:int}{out:int}
Takes a number of inputs which are usually zero. If any of the
inputs are nonzero, the output will be 1. Typically used to combine
a number of clocks (or similar) into a single stream.

\comp{osc}
{Generates a periodic waveform.}
{mod:float}{wave:float}
Given a frequency in periods/beat (according to the current tempo),
produces a wave on the output. Various waveforms are available.
The phase can be modulated by an input.
\startparams
wave & enum & waveform\\
freq & float & frequency (periods/beat)\\
phase & float & phase\\
mod & float & amount of phase modulation\\
amp & float & amplitude of wave\\
offset & float & added to output\\
minzero & boolean & if true, wave's default range becomes $[0,amp]$ rather
than $[-amp,amp]$\\
width & float & pulsewidth if square wave selected (default is 0.5, which is a true square wave)
\end{tabularx}


\comp{time}
{Gives the time since the program started.}
{}{time:float}
Produces the number of seconds since the start of the program,
times a ``rate'' parameter.
\startparams
rate & float & the number of seconds is multiplied by this to
produce the output
\end{tabularx}

\clearpage
\comp{env}{Envelope generator}
{trigger:int}{out:float}
This is a 6 stage envelope generator. Each stage has a level
and a time relative to the start, in seconds. There is an implied
level=0, time=0 stage. The generator can function in three modes:
\begin{itemize}
\item \textbf{retrig}: the envelope restarts whenever \emph{trig} is
non-zero;
\item \textbf{noretrig}: the envelope only restarts when \emph{trig} is
high and any previous envelope has completed;
\item \textbf{freerun}: the envelope retriggers as soon as it stops
running, and starts running as soon as its output is connected to
a running component.
\end{itemize}
\startparams
mode & enum & retrigger mode as described above\\
rate powerof2 & int & the time since the start is multiplied by 2
to this power before being used to calculate the envelope state\\
env & env & the actual envelope (considered as a single parameter
so the editor can put a fancy graphic on it)
\end{tabularx}
Within the envelope parameter each stage has a level and time.
Times are autocorrected: a later stage cannot have its time moved
to before an earlier stage, and moving an earlier stage to after
a later stage will cause the later stage times to move.
Two stages may have the same time, in which case the higher-numbered
stages level is jumped to.

\comp{clockdiv}{Clock divider}
{tick:int}{tick:int}
The clock divider accepts a sequence of ticks, and outputs every $n$th one.
\startparams
div & int & clock division ratio -- output 1 tick for every $div$ input ticks
\end{tabularx}

\section{State modifier components}
All these components work by modifying some global state, so that
the components below them in the tree use this modified value.
On exit from the component, the original state is restored.

See also the \emph{tempo} component (p.~\pageref{comp:tempo}).

\comp{vel}
{Modifies note velocity for components below}
{in:4$\times$input:flow, mod:float}{output:flow}
Modify the velocity of all notes played by components below
this. The new velocity will be
\[
vel \gets vel \times (p_{vel}+p_{mod}\times in)
\]
\startparams
vel & float & the amount to multiply the velocity by\\
mod & float & the amount to multiply the modulation input by, which
is then added to $vel$
\end{tabularx}


\section{Mathematics components}
\comp{add}
{Adds the inputs together}
{in1:float,in2:float}{$mul1\times(in1+add1) + mul2\times(in2+add2)$}
Adds together the inputs, after they have been processed through
some parameters.
\startparams
add1 & float & added to input 1 before multiplication by mul1\\
add2 & float & added to input 2 before multiplication by mul2\\
mul1 & float & multiplies $in1+add1$\\
mul2 & float & multiplies $in2+add2$\\
\end{tabularx}

\comp{mul}
{Multiplies the inputs together.}
{in1:float,in2:float}{$mul1\times(in1+add1) \times mul2\times(in2+add2)$}
Multiplies together the inputs, after they have been processed through
some parameters. Yes, the two multiplies are redundant.
\startparams
add1 & float & added to input 1 before multiplication by mul1\\
add2 & float & added to input 2 before multiplication by mul2\\
mul1 & float & multiplies $in1+add1$\\
mul2 & float & multiplies $in2+add2$\\
\end{tabularx}

\clearpage
\comp{func}
{Performs one of a number of functions on the sum of the inputs.}
{in1:float,in2:float}{$addout+mulout\times f \big(mul1(in1+add1)+ mul2(in2+add2)\big)$}
Takes the two inputs and processes them through multiply and add
parameters, puts the result through a function, and then puts the
output through another multiply and add stage. If an input is
not connected, it is zero.
Functions supported include: sine, cosine, gaussian(x+y), x mod y, abs(x+y)
\startparams
add1 & float & added to input 1 before multiplication by mul1\\
mul1 & float & multiplies $in1+add1$\\
add2 & float & added to input 2 before multiplication by mul2\\
mul2 & float & multiplies $in2+add2$\\
func & enum & function to perform\\
mulout & float & multiplies function output\\
addout & float & added to output
\end{tabularx}

\section{Music}
These components form the largest group, and deal with
actually generating and playing music. This may get split into
subgroups.

\comp{chord}
{Generate a chord, whose notes can be selected and played by
\emph{chordplay} or \emph{noteplay} etc.}
{base:float, select:float}{chord:chord}
This component is used to generate a set of notes -- a chord -- given
a base note and a chord selector.
The output chord is an unordered set of MIDI note values which can
be played by other components.

The base note is a note number where 0 is C4. The chord
has a scale, which is selected by a parameter. There are then four
possible ``chord string'' parameters, which are selected from by
the \emph{select} input. Each string is a number of hex digits,
each of which is an offset into the scale. Thus, the string ``024''
when the scale is ``major'', defines a major triad:
the I chord off the base note. The string ``468'' will define the V
chord, and ``0246'' defines a Vmaj7 chord\footnote{a V7 chord would
require the minor 7th interval, which is not in the scale. I'm working
on ways around this!}.

The \emph{octave} parameter (times 12) is added to the resulting 
notes.
If the \emph{inversions} parameter is true, the new chord is compared with
the previous played chord and the inversion which gives the notes
closest to the previous chord is used.
\startparams
default chord & string & the chord string used when $\lfloor select \rfloor$ is zero or unconnected \\
chord 1 & string & the chord string used when $\lfloor select \rfloor=1$\\
chord 2 & string & the chord string used when $\lfloor select \rfloor=1$\\
chord 3 & string & the chord string used when $\lfloor select \rfloor=1$\\
octave & int & the octave number (0=C4-B$^\#$4)\\
inversions & boolean & if true, use an inversion which makes the notes
as close to the previous chord as possible\\
print & boolean & if true, print the resulting notes to stdout\\
scale & enum & the scale to select the notes from
\end{tabularx}

\clearpage
\comp{chordplay}
{Play all the notes in a chord generated by e.g. the \emph{chord} component.}
{chord:chord, tick:int, velmod:float, durmult:float}{out:flow}
This component takes a chord and plays all its notes to a given MIDI
channel at the same time, or with a short gap between them from low to
high (to give
a rapid arpeggio). The duration of the chord can be specified, the
velocity modified, and the chord transposed.
The chord will start playing when \emph{tick} is non-zero -- this
signal can be generated by a \emph{clock} (p.\pageref{comp:clock}).
The actual velocity is
\[
v = g_{vel} \times \Big(p_{vel}+(p_{velmod}\times in_{velmod})\Big)
\]
and the actual duration is
\[
d = p_{dur} \times in_{durmult} \times 2^{p_{durpow2}}
\]
Note that if the \emph{durmult} input is not connected, its default
value is 1.
\startparams
channel & int & MIDI channel \\
gapsecs & float & gap to leave between subsequent notes in the chord
(low to high) in seconds\\
vel & float & velocity (is multiplied by global velocity)\\
velmod & float & the \emph{vel} input is multiplied by this, then
added to the \emph{vel} parameter.\\
duration & float & base duration (beats)\\
duration-pow2 & float & duration is multiplied by 2 to this power\\
transpose & int & interval in semitones, added to the output note
\end{tabularx}

\clearpage
\comp{noteplay}
{When triggered, plays a note selected from a chord by an input.}
{gate:int, trig:int, note:float, velmod:float, chord:chord, durmult:float}
{out:flow}
When \emph{trig} is nonzero and \emph{gate} is nonzero or unconnected,
this component uses $\lfloor note \rfloor$ to select a note from a chord
and plays it. Typically, the \emph{note}, \emph{trig} and \emph{durmult} inputs
come from a sequencer such as \emph{denseq}.
As with \emph{chordplay}, the velocity and duration are
\[
v = g_{vel} \times \Big(p_{vel}+(p_{velmod}\times in_{velmod})\Big)
\]
and the actual duration is
\[
d = p_{dur} \times in_{durmult} \times 2^{p_{durpow2}}
\]
Note that if the \emph{durmult} input is not connected, its default
value is 1.
\startparams
channel & int & MIDI channel \\
vel & float & velocity (is multiplied by global velocity)\\
velmod & float & the \emph{vel} input is multiplied by this, then
added to the \emph{vel} parameter.\\
duration & float & base duration (beats)\\
duration-pow2 & float & duration is multiplied by 2 to this power\\
transpose & int & interval in semitones, added to the output note\\
suppress retrig & bool & if true, do not allow notes to be played
if this player is already playing them (some other player could do it,
though)
\end{tabularx}

\clearpage
\comp{denseq}{Denary sequencer}
{tick:int}{eventtick:int, value:float, cycletick:int, cyclecount:float,
eventduration:float}
This component uses a string of binary digits to generate numbers
which can be used to control components such as \emph{noteplay}.
The string can also contain gaps (indicated by ``-'' or spaces)
which increase the length of the preceding event.
Each time \emph{tick} is true, the sequencer advances one step.
If this step takes it to a new event, it will
\begin{itemize}
\item set \emph{eventtick} high,
\item set\emph{value} to the digit in the string (0-9),
\item set \emph{cycletick} to high and increment \emph{cyclecount} if
the tick returned us to the beginning,
\item set \emph{eventduration} to the duration of the event (1+the number
of subsequent gaps).
\end{itemize}
The sequencer will loop forever, as long as ticks keep arriving.
Some examples:
\begin{itemize}
\item ``012'' will simply produce the values 0, 1 and 2 with a duration
of 1 on every tick. Every three ticks, \emph{cycletick} will spike high,
and \emph{cyclecount} will increase.
\item ``01 2'' or ``01-2'' will produce the following sequence of outputs:

\begin{tabular}{llllll}
eventtick & value & eventduration & cycletick & cyclecount\\
1 & 0 & 1 & 0 & 0 & event of length 1, output 0\\
1 & 1 & 2 & 0 & 0 & event of length 2, output 1\\
0 & 1 & 2 & 0 & 0 & no event (still doing the previous one)\\
1 & 2 & 1 & 0 & 0 & event of length 1, output 2\\
1 & 0 & 1 & 1 & 1 & cycle repeats\\
1 & 1 & 2 & 0 & 1 & \\
0 & 1 & 2 & 0 & 1 & \\
1 & 2 & 1 & 0 & 1 &
\end{tabular}
\end{itemize}
A typical usage of \emph{denseq} is to wire it to a \emph{noteplay} 
thus:

\begin{tabular}{lll}
denseq & & noteplay\\
\emph{eventtick} & $\to$ & \emph{notetrig} \\
\emph{value} & $\to$ & \emph{note} \\
\emph{eventduration} & $\to$ & \emph{durmult} 
\end{tabular}
\startparams
seq & string & sequence of denary digits and gap characters defining the
output values and durations
\end{tabularx}

\clearpage
\comp{denpick}{Pick values from denary string (can be random)}
{tick:int,select:float}{eventtick:int,value:float}
This component has a string of denary digits (similar to \emph{denseq} 
but gaps are not permitted). When the input ticks, it sets the
output value to a value chosen from this string by \emph{select}.
If there is no \emph{select} input, the value is chosen randomly.
\startparams
vals & string & denary values to pick from \\
unit in & boolean & if true, the input is in the range 0-1 and is
scaled up to the full range. Otherwise $\lfloor select \rfloor$ 
selects the value. Has no effect if \emph{select} is not connected.\\
cycle & boolean & if \emph{select} is not connected, i.e. random
selection is used, will shuffle the numbers each cycle and output
them one at a time rather than just picking a random element each time.
This produces a better distribution.
\end{tabularx}

\section{Generators}
Generators use algorithms to generate sequences of data, as opposed
to \emph{denseq} and \emph{denpick} which just generate data
in sequence or pick from a sequence.

\comp{thue-morse}
{Generate data using the Generalized Thue-Morse Sequence}
{tick:int,gate:float,regen:int}
{eventtick:int,output:float,cycletick:int,cyclecount:float}
This component generates pseudorandom sequences using a Thue-Morse
algorithm. The \emph{length} parameter determines how long the sequence
should be, and the \emph{depth} parameter determines how many different
values are picked. The \emph{perms} are a string of digits, which
are the values which are set in the actual output from the 
output of the algorithm, at each stage. Changing these changes the
pattern in the output, but don't expect to see values higher than
\emph{depth} even if you set them in \emph{perms}.

The \emph{regen} input will cause the generator to shuffle its
permutations and restart, giving a new sequence. This can be
done automatically with a parameter.

\textbf{Note} that the \emph{gate} input will not only suppress the
output tick, but it will also stop the next note being selected. This
seems obvious, but it is tempting to feed the output of a Thue-Morse
directly into the gate of another to produce rests. This won't work
(or rather it will, but not as you expect). Instead, to generate
rests:
\begin{itemize}
\item create a note-generating TM.
\item create a gate-generating TM with a depth set to the chance
of a rest happening -- so (say) 10 for a 1 in 10 chance.
\item Connect the trigger and output of the first into a \emph{noteplay}.
\item Connect the output of the second to the gate of the \emph{noteplay}.
\item Connect both TMs to the same trigger.
\end{itemize}
\startparams
length & int & length of sequence\\
depth & int & how many different values are generated $(0\cdots depth-1)$\\
perms & string & a string of hex digits for the permutations --
all that matters is the relative frequencies, they are shuffled and
added to and modded with the depth. Use it as a random seed for the 
sequence\\
symmetric & bool & generates a sequence whose second half is the first
half backwards\\
auto-regen & bool & automatically create a new sequence when the old
one ends
\end{tabularx}

\section{Data connection components}
\comp{diamond}
{Receives data from \emph{Diamond Apparatus} topics}
{}{out:float}
Receives data from a single floating point topic on Diamond Apparatus,
and outputs it.
\startparams
topic & string & name of the topic\\
mul & float & topic is multiplied by this\\
add & float & added to the multiplied result
\end{tabularx}

\comp{OSC-out}
{Sends data over Open Sound Control protocol}
{in$\times$6:float}{flow:flow}
Sends data over OSC (Open Sound Control) as a float array.
Typically this is to a SuperCollider program, so the port
is set to 57120. Only connected ports are sent, so if three
ports are connected the message will be three floats. A message
is only sent if one of the inputs has changed value since last
time.
\startparams
topic & string & name of the topic (e.g. \texttt{/foo})\\
\end{tabularx}



\end{document}

