\section{Control components}
\comp{output}
{A ``root'' component which simply runs all its inputs}
{4$\times$input:flow}{}
The patch runs by finding all \emph{output} components and running
their inputs. Therefore, this component must be present for the patch to run. 

\comp{mixer}
{A component which simply runs all its inputs}
{6$\times$input:flow}{output:flow}
This component, when run, runs all its inputs.

\comp{switcher}
{Selects one input to pass through to the output}
{select:float, 6$\times$inputs:ANY}{output:ANY}
This component simply selects one of its inputs and executes it,
passing the obtained value to the output without interpreting it.
The output and inputs (apart from the selection value) may therefore
be of any type. 

The selection value is typically in the range [0,1], and is
mapped onto [0,$n$] where $n$ is the number of the highest connected
\emph{any} input. For example, if only two \emph{any} inputs are connected,
values less than 0.5 will select the first choice, otherwise the 
second choice will be selected.
\startparams
mul & float & select input is multiplied by this value before use \\
add & float & select input is then added to this value before use\\
unit & bool & determines whether or not the value is mapped onto [1,0]
or [0,$n-1$] for selection (true means the former)
\end{tabularx}

\clearpage
\comp{threshold}
{Outputs triggers when thresholds are crossed}
{input:float}{trigger$\times$6:int}
Contains 6 thresholding units, which detect when the input
crosses a boundary (rising or falling) and output a single pulse
trigger on the output for the unit.
\startparams
mul & float & input is multiplied by this value before use \\
add & float & input is then added to this value before use\\
&&\\
level $n$ & float & level of threshold\\
type $n$     & inactive,rising or falling & threshold type\\
\end{tabularx}
