\section{Mathematics components}
\comp{const}
{Output a constant}
{}{out:float}
Outputs a constant value.
\startparams
val & float & value to output\\
\end{tabularx}

\comp{add}
{Adds the inputs together}
{in1:float,in2:float}{$mul1\times(in1+add1) + mul2\times(in2+add2)$}
Adds together the inputs, after they have been processed through
some parameters.
\startparams
add1 & float & added to input 1 before multiplication by mul1\\
add2 & float & added to input 2 before multiplication by mul2\\
mul1 & float & multiplies $in1+add1$\\
mul2 & float & multiplies $in2+add2$\\
\end{tabularx}

\comp{mul}
{Multiplies the inputs together.}
{in1:float,in2:float}{$mul1\times(in1+add1) \times mul2\times(in2+add2)$}
Multiplies together the inputs, after they have been processed through
some parameters. Yes, the two multiplies are redundant.
\startparams
add1 & float & added to input 1 before multiplication by mul1\\
add2 & float & added to input 2 before multiplication by mul2\\
mul1 & float & multiplies $in1+add1$\\
mul2 & float & multiplies $in2+add2$\\
\end{tabularx}

\clearpage
\comp{func}
{Performs one of a number of functions on the sum of the inputs.}
{in1:float,in2:float}{$addout+mulout\times f \big(mul1(in1+add1)+ mul2(in2+add2)\big)$}
Takes the two inputs and processes them through multiply and add
parameters, puts the result through a function, and then puts the
output through another multiply and add stage. If an input is
not connected, it is zero. Thus, the output is:
\[
p_{addout} + p_{mulout} \times f\Big((p_{add1}+in_1)\times p_{mul1} +
(p_{add2}+in_2)\times p_{mul2} \Big)
\]
Functions supported include: sine, cosine, gaussian(x+y), x mod y, abs(x+y).
\startparams
add1 & float & added to input 1 before multiplication by mul1\\
mul1 & float & multiplies $in1+add1$\\
add2 & float & added to input 2 before multiplication by mul2\\
mul2 & float & multiplies $in2+add2$\\
func & enum & function to perform\\
mulout & float & multiplies function output\\
addout & float & added to output
\end{tabularx}

\clearpage
\comp{perlin}{Perlin coherent noise generator (input is values or current time)}
{x:float,y:float}{out:perlin(x+y),zerocrosstrig:int}
Calculates 
\[
p_{addout} + p_{mulout} \times \operatorname{perlin}\Big((p_{add1}+in_1)\times p_{mul1} +
(p_{add2}+in_2)\times p_{mul2} \Big)
\]
where \emph{perlin} is the Perlin noise function with range [-1,1].
Thus, it generates
smoothly varying coherent random noise. If neither input is connected, 
$in_1$ is the current time. If the output crosses zero, i.e. the previous
output was one side of zero and the new output is the other side, 
the \emph{zerocrosstrig} output is set high. This is typically used
with no inputs to generate randomly timed events, with \emph{mul1} set
to control the frequency.
\startparams
octaves & int & number of octaves in noise function -- the higher, the
more ``fractal'' the noise and the ``busier'' it becomes\\
persistence & float & the noise persistence -- how much the amplitudes
decrease for each successive octave\\
add1 & float & added to input 1 (or time if no inputs are connected) 
before multiplication by mul1\\
mul1 & float & multiplies $in1+add1$\\
add2 & float & added to input 2 before multiplication by mul2\\
mul2 & float & multiplies $in2+add2$\\
mulout & float & multiplies function output\\
addout & float & added to output
\end{tabularx}
For more information on \emph{octaves} and \emph{persistence}, and Perlin
noise is general, see the documentation for \texttt{libnoise}.
