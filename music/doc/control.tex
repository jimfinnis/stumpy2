\section{Control components}
\comp{output}
{A ``root'' component which simply runs all its inputs}
{4$\times$input:flow}{}
The patch runs by finding all \emph{output} components and running
their inputs. Therefore, this component must be present for the patch to run. 

\comp{mixer}
{A component which simply runs all its inputs}
{6$\times$input:flow}{output:flow}
This component, when run, runs all its inputs.

\comp{switcher}
{Selects one input to pass through to the output}
{select:float, 6$\times$inputs:ANY}{output:ANY}
This component simply selects one of its inputs and executes it,
passing the obtained value to the output without interpreting it.
The output and inputs (apart from the selection value) may therefore
be of any type. 

The selection value is typically in the range [0,1], and is
mapped onto [0,$n$] where $n$ is the number of the highest connected
\emph{any} input. For example, if only two \emph{any} inputs are connected,
values less than 0.5 will select the first choice, otherwise the 
second choice will be selected.
\startparams
mul & float & select input is multiplied by this value before use \\
add & float & select input is then added to this value before use\\
unit & bool & determines whether or not the value is mapped onto [1,0]
or [0,$n-1$] for selection (true means the former)
\end{tabularx}

\clearpage
\comp{threshold}
{Outputs triggers when thresholds are crossed}
{input:float}{trigger$\times$6:int}
Contains 6 thresholding units, which detect when the input
crosses a boundary (rising or falling) and output a single pulse
trigger on the output for the unit; or alternatively a 1 or 0
depending on whether the value is below or above threshold.
\startparams
mul & float & input is multiplied by this value before use \\
add & float & input is then added to this value before use\\
&&\\
level $n$ & float & level of threshold\\
type $n$     & inactive, rising, falling, low, high& threshold type\\
\end{tabularx}


\comp{connectin}
{Connector (input)}
{input:ANY}{}
This is the input side of a connector, which can be used to
organise complex patches. The output comes out of an output
connector with the same number.


\comp{connectout}
{Connector (output)}
{output:ANY}{}
This is the output side of a connector, which can be used to
organise complex patches. The input comes from an input
connector with the same number.

\comp{runslow}
{Run components occasionally (useful for control data send,
such as midi CC)}
{input:$\times 6$:flow}{output:flow}
This will only run the flow inputs if the current number of
update ticks is a multiple of the \emph{steps} parameter.
\startparams
interval & int & interval at which to run inputs
\end{tabularx}

\clearpage
\comp{crossfade}
{Crossfader}
{select:float, input$\times$6:flow}{output:flow}
This changes the amplitude of the inputs depending on which one
is selected by modifying the global amplitude. 
Inputs whose amplitude is nonzero will not run at all.
The select value determines which input is ``loudest''.
If the select value is an integer, only that input
will have a nonzero amplitude. If the select value is between
two inputs (it is a float), the two inputs either side will both
run, with their relative amplitudes dependent on how far between
them the select value is.
\[
amp(n) = \begin{cases}
1-|sel-n| & \quad\mathrm{if}\, 1-|sel-n|>0\\
0 & \quad\mathrm{otherwise}
\end{cases}
\]

