\section{Mathematics components}
\comp{add}
{Adds the inputs together}
{in1:float,in2:float}{$mul1\times(in1+add1) + mul2\times(in2+add2)$}
Adds together the inputs, after they have been processed through
some parameters.
\startparams
add1 & float & added to input 1 before multiplication by mul1\\
add2 & float & added to input 2 before multiplication by mul2\\
mul1 & float & multiplies $in1+add1$\\
mul2 & float & multiplies $in2+add2$\\
\end{tabularx}

\comp{mul}
{Multiplies the inputs together.}
{in1:float,in2:float}{$mul1\times(in1+add1) \times mul2\times(in2+add2)$}
Multiplies together the inputs, after they have been processed through
some parameters. Yes, the two multiplies are redundant.
\startparams
add1 & float & added to input 1 before multiplication by mul1\\
add2 & float & added to input 2 before multiplication by mul2\\
mul1 & float & multiplies $in1+add1$\\
mul2 & float & multiplies $in2+add2$\\
\end{tabularx}

\clearpage
\comp{func}
{Performs one of a number of functions on the sum of the inputs.}
{in1:float,in2:float}{$addout+mulout\times f \big(mul1(in1+add1)+ mul2(in2+add2)\big)$}
Takes the two inputs and processes them through multiply and add
parameters, puts the result through a function, and then puts the
output through another multiply and add stage. If an input is
not connected, it is zero.
Functions supported include: sine, cosine, gaussian(x+y), x mod y, abs(x+y)
\startparams
add1 & float & added to input 1 before multiplication by mul1\\
mul1 & float & multiplies $in1+add1$\\
add2 & float & added to input 2 before multiplication by mul2\\
mul2 & float & multiplies $in2+add2$\\
func & enum & function to perform\\
mulout & float & multiplies function output\\
addout & float & added to output
\end{tabularx}
