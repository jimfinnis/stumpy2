\section{Generators}
Generators use algorithms to generate sequences of data, as opposed
to \emph{denseq} and \emph{denpick} which just generate data
in sequence or pick from a sequence.

\comp{thue-morse}
{Generate data using the Generalized Thue-Morse Sequence}
{tick:int,gate:float,regen:int}
{eventtick:int,output:float,cycletick:int,cyclecount:float}
This component generates pseudorandom sequences using a Thue-Morse
algorithm. The \emph{length} parameter determines how long the sequence
should be, and the \emph{depth} parameter determines how many different
values are picked. The \emph{perms} are a string of digits, which
are the values which are set in the actual output from the 
output of the algorithm, at each stage. Changing these changes the
pattern in the output, but don't expect to see values higher than
\emph{depth} even if you set them in \emph{perms}.

The \emph{regen} input will cause the generator to shuffle its
permutations and restart, giving a new sequence. This can be
done automatically with a parameter.

\textbf{Note} that the \emph{gate} input will not only suppress the
output tick, but it will also stop the next note being selected. This
seems obvious, but it is tempting to feed the output of a Thue-Morse
directly into the gate of another to produce rests. This won't work
(or rather it will, but not as you expect). Instead, to generate
rests:
\begin{itemize}
\item create a note-generating TM.
\item create a gate-generating TM with a depth set to the chance
of a rest happening -- so (say) 10 for a 1 in 10 chance.
\item Connect the trigger and output of the first into a \emph{noteplay}.
\item Connect the output of the second to the gate of the \emph{noteplay}.
\item Connect both TMs to the same trigger.
\end{itemize}
\startparams
length & int & length of sequence\\
depth & int & how many different values are generated $(0\cdots depth-1)$\\
perms & string & a string of hex digits for the permutations --
all that matters is the relative frequencies, they are shuffled and
added to and modded with the depth. Use it as a random seed for the 
sequence\\
symmetric & bool & generates a sequence whose second half is the first
half backwards\\
auto-regen & bool & automatically create a new sequence when the old
one ends
\end{tabularx}
