\section{State modifier components}
All these components work by modifying some global state, so that
the components below them in the tree use this modified value.
On exit from the component, the original state is restored.

See also the \emph{tempo} component (p.~\pageref{comp:tempo}).

\comp{vel}
{Modifies note velocity for components below}
{in:4$\times$input:flow, mod:float}{output:flow}
Modify the velocity of all notes played by components below
this. The new velocity will be
\[
vel \gets vel \times (p_{vel}+p_{mod}\times in)
\]
\startparams
vel & float & the amount to multiply the velocity by\\
mod & float & the amount to multiply the modulation input by, which
is then added to $vel$
\end{tabularx}

\comp{transposer}
{Transposes notes in components below}
{in:4$\times$input:flow, mod:float}{output:flow}
This transposes all notes played by components below it in the tree.
\startparams
trans & int & the number of semitones to transpose by\\
oct & int & the number of octaves to transpose by
\end{tabularx}

\comp{chordstate}
{Set the global default chord}
{in:4$\times$input:flow, chord}{output:flow}
This component sets the chord used by components which have a single
chord input when that input is unconnected, i.e. the default chord.
Such components include \emph{noteplay} and \emph{chordplay}. All
components which are below this component in the tree will use this
chord as the default chord.
